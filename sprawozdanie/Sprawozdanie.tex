\documentclass[a4paper]{article}
\usepackage[T1]{fontenc} % Polskie znaki
\usepackage{graphicx} % Wstwianie grafiki
\usepackage{subcaption}
\usepackage{float}
%opening
\title{Sprawozdanie\\Projekt Sieci Komputerowe 2}
\author{Magdalena Wiechczyńska, 132337}
\date{} %Usunięcie daty

\begin{document}

\maketitle

\section{Protokół komunikacyjny}
	Serwer po uruchomieniu tworzy wątek, który odpowiada za łączenie oczekujących na grę graczy w pary. Dla wybranej pary deskryptorów klientów tworzony jest nowy wątek obsługujący ich grę.
	
	W pierwszej kolejności wątek gry prosi po kolei klientów o ich loginy, wysyłając wiadomość \textit{LOGIN}.
Klienci odsyłają swoje loginy, które muszą być zakończone znakiem nowej linii. Następnie serwer przesyła klientom odpowiednio \textit{white} oraz \textit{black}, które powiadamiają ich o kolorze ich bierek. Jest to równoznaczne z pomyślnym rozpoczęciem gry i serwer będzie teraz oczekiwał na ruch gracza białego.

	Wiadomość zawierająca ruch składa się z 5 znaków, prezentuje dwa pola oddzielone znakiem ':', gdzie pierwsze pole to pozycja startowa ruszanej bierki, a drugie to pole na którym kończy swój ruch.
Przykład poprawnie zapisanego ruchu: \textit{E2:E4}. Serwer weryfikuje, czy jest to prawidłowy ruch. Jeżeli wiadomość nie jest poprawna składniowo, serwer odsyła nadawcy wiadomość \textit{E:ASC} (od ASCII), po czym raz jeszcze oczekuje na podanie prawidłowego ruchu. Jeżeli na polu startowym nie istnieje żadna bierka gracza, odsyłany jest komunikat \textit{E:STA} (od słowa start). Jeżeli bierka istnieje, lecz nie jest możliwe wykonanie zadanego ruchu, to odsyłany jest komunikat \textit{E:KON}. (od słowa koniec). Gdy ruch jest prawidłowy, serwer aktualizuje swój stan gry oraz sprawdza, czy przeciwnik ma jakikolwiek możliwy ruch. Jeżeli tak, to odsyła mu wykonany ruch oraz oczekuje na jego.
Jeżeli nie, oznacza to, że nastąpił szach-mat.
	
	Zamiast ruchu przeciwnika, serwer może przesłać klientowi wiadomość \textit{\_INFO}. Klient wie wtedy, że zaszło pewne zdarzenie i powinien odebrać kolejne wiadomości serwera przed wykonaniem swojego ruchu. Jeżeli przeciwnik rozłączył się z serwerem, gracz otrzyma od serwera informację \textit{\_CONN}. Jeżeli nastąpił szach-mat, gracz otrzyma wiadomość o treści \textit{WIN:W} albo \textit{WIN:B}, w zależności od koloru wygrywającego.

	Po rozłączeniu jednego z graczy bądź po macie wątek gry jest niszczony.
	
\section{Struktura projektu}
    Projekt składa się z programu serwera, napisanego w języku C++ oraz z programu klienta, napisanego w języku Java.

	Kod programu serwera znajduje się w jednym pliku .cpp. Program klienta składa się z trzynastu klas, z których 9 odpowiada za logikę gry, 2 to kontrolery okien GUI, jedna służy do komunikacji z serwerem a ostatnia to klasa Main, w której inicjowane są wszystkie elementy GUI. 
    
\section{Sposób kompilacji i uruchomienia projektu}
	Serwer kompiluje się bez błędów wykonując komendę \textit{g++ -pthread -o serwer server.cpp}. Przy uruchomieniu serwera należy pamiętać, że wymagane jest podanie numeru portu serwera jako argument, np. \textit{./serwer 1234}
	
	Program klienta najlepiej uruchomić z poziomu IDE Intelli. Aby uruchomić kilka instancji klienta na jednym urządzeniu, należy w oknie Run/Debug Configurations zaznaczyć opcję Allow parallel run.


\end{document}
